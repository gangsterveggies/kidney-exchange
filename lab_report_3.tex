%----------------------------------------------------------------------------------------
%	CONFIGURATIONS
%----------------------------------------------------------------------------------------

\documentclass[10pt,a4paper,oneside]{article}

\usepackage[utf8]{inputenc}
\usepackage{verbatim}
\usepackage{graphicx}
\usepackage{natbib}
\usepackage{amsmath}
\usepackage{lipsum}
\usepackage{caption}
\usepackage{subcaption}
\usepackage{fontenc}
\usepackage[utf8]{inputenc}
\usepackage{mathtools}
\usepackage[a4paper,left=2cm,right=2cm,top=1.5cm,bottom=2.5cm]{geometry}

%----------------------------------------------------------------------------------------
%	INFORMATION
%----------------------------------------------------------------------------------------

\title{Métodos de Apoio à Decisão - Trabalho 3}

\author{João Ramos\footnote{João Ramos -  201204672} e Pedro Paredes\footnote{Pedro Paredes - 201205725}, DCC - FCUP}

\date{Maio 2015}

\renewcommand\tablename{Tabela}

\begin{document}

\maketitle

%----------------------------------------------------------------------------------------
%	SECTION 0
%----------------------------------------------------------------------------------------

\section*{Introdução}

Neste relatório descrevem-se as abordagens e resultados ao problema
enunciado no trabalho 3 com os vários métodos descritos no enunciado.

Todas as soluções foram implementadas na linguagem \texttt{Python},
usando adicionalmente as seguintes conhecidas bibliotecas externas:
\texttt{numpy}, \texttt{scipy} e \texttt{networkx}. O tipo de output
de cada implementação é semelhante, mudando apenas os métodos base a
usar.

Anexo ao relatório estão 4 ficheiros de \texttt{Python} e 4 ficheiros
\texttt{txt} com as soluções produzidas por cada um. O ficheiro
\texttt{fifo\_simulation.py} corresponde à solução usando o método da
alínea 1. O ficheiro \texttt{fifo\_simulation2.py} corresponde à
solução usando o método da alínea 2. O ficheiro
\texttt{mwm\_simulation.py} corresponde à solução usando o método da
alínea 3. Finalmente, o ficheiro
\texttt{mwm\_simulation\_alternative.py} corresponde a uma solução
alternativa baseada no método da alínea 3, cuja descrição é feita numa
secção futura.

O resto do relatório está organizado da seguinte forma. A Secção
\ref{sec:model1} descreve o modelo descrito na alínea 1 e os seus
resultados. A Secção \ref{sec:model2} descreve o modelo descrito na
alínea 2 e os seus resultados. A Secção \ref{sec:model3} descreve o
modelo descrito na alínea 3 e os seus resultados, assim como algumas
alternativas. Finalmente, a Secção \ref{sec:limits} descreve algumas
das limitações do modelo assim como hipóteses que tiveram de ser
assumidas, complementando a resposta a cada uma das alíneas do
enunciado.
 
%----------------------------------------------------------------------------------------
%	SECTION 1
%----------------------------------------------------------------------------------------

\section{Modelo FIFO Inicial}
\label{sec:model1}

O modelo simula a componente temporal do problema usando uma
estratégia de \textit{next-event time advance}. Em cada iteração da
simulação, é feita uma amostra da distribuição exponencial, usando os
parâmetros de média descritos no enunciado, para determinar o tempo em
que o próximo par entrará na fila e é gerado um par dador-paciente
fazendo uma amostra de uma distribuição uniforme, de modo a respeitar
as percentagens de ocorrência cada grupo sanguíneo na população
portuguesa indicadas. A unidade de tempo considerada é a semana, sendo
a mais apropriada de acordo com os valores do enunciado.

Após obter cada par, são colocados na fila de espera, mas caso o dador
seja compatível com o paciente do par, é logo retirado da fila
(considerando que é feito um transplante entre os membros do
par). Imediatamente é feita uma verificação se existe algum outro par
que seja compatível com o recentemente inserido. Caso exista algum
outro par compatível, então é feito um transplante cruzado entre os
dois e ambos são retirados da fila de espera. Este processo é repetido
até o tempo total da simulação chegar às 156 semanas (3 anos).

O programa recolhe um conjunto de estatísticas em 1000 simulações
realizadas, nomeadamente: a média de pares na fila de espera por
semana, assim como um intervalo de confiança de 95\% para este valor;
a média de pares na fila de espera no final da simulação, assim como
um intervalo de confiança de 95\% para este valor; a média do número
de dadores e de pacientes por cada grupo sanguíneo diferente; uma
matriz de número de transplantes médio entre cada grupo sanguíneo,
assim como o número total de transplantes e a mesma matriz normalizada
pelo número total de transplantes.

Os resultados obtidos por este modelo estão contidos no ficheiro
\texttt{fifo\_simulation.txt}. A Tabela \ref{tab:t1} apresenta um
sumário destes resultados.

\begin{table}[h]
  \centering
  \begin{subtable}[h]{0.3\textwidth}
    \centering
    45
    \caption{Média de pares na fila de espera por semana}
  \end{subtable}
  ~
  \begin{subtable}[h]{0.3\textwidth}
    \centering
    86
    \caption{Média de pares na fila de espera no final da simulação}
  \end{subtable}
  ~
  \begin{subtable}[h]{0.3\textwidth}
    \centering
    $226.28$
    \caption{Número total de transplantes}
  \end{subtable}

  \caption{Resultados da simulação}
  \label{tab:t1}
\end{table}

Um problema que foi observado nos resultados e que acabou por
introduzir alguma variação nos valores obtidos, é o facto de que
alguns pares possíveis nunca saírem da fila de espera, nomeadamente os
que têm pacientes do tipo \texttt{O} e dadores de outro tipo. Assim,
estes tipos de par vão acumulando na fila de espera, aumentando tanto
a média, como o número final de pares na fila de espera.

Observando os resultados detalhados no ficheiro indicado, é possível
notar que existe uma boa percentagem de transplantes que são feitos de
dadores e pacientes do mesmo par, um resultado que foi confirmado por
outras experiências feitas. Como seria de esperar pela percentagem de
ocorrências de cada grupo sanguíneo, resultam numa maioria dos
transplantes onde participa um indivíduo do tipo \texttt{O} ou
\texttt{A}. Finalmente, é também de notar que os resultados são
compatíveis com os parâmetros dados no enunciado, por exemplo, a média
de chegada de pares por semana é de aproximadamente 2.


%----------------------------------------------------------------------------------------
%	SECTION 2
%----------------------------------------------------------------------------------------

\section{Modelo FIFO Corrigido}
\label{sec:model2}

Neste modelo corrigido, é utilizado como base o modelo da Secção
\ref{sec:model1}. É introduzida uma correção quando é feita a
comparação de compatibilidade, tanto para o caso dos elementos do par
serem compatíveis considerando apenas o grupo sanguíneo como quando é
feita a comparação do par com os restantes pares já na fila de
espera.

Os valores medidos são equivalentes aos medidos na secção anterior e a
sua descrição é feita de modo semelhante.

Os resultados obtidos por este modelo estão contidos no ficheiro
\texttt{fifo\_simulation2.txt}. A Tabela \ref{tab:t2} apresenta um
sumário destes resultados.

\begin{table}[h]
  \centering
  \begin{subtable}[h]{0.3\textwidth}
    \centering
    42
    \caption{Média de pares na fila de espera por semana}
  \end{subtable}
  ~
  \begin{subtable}[h]{0.3\textwidth}
    \centering
    78
    \caption{Média de pares na fila de espera no final da simulação}
  \end{subtable}
  ~
  \begin{subtable}[h]{0.3\textwidth}
    \centering
    $235.43$
    \caption{Número total de transplantes}
  \end{subtable}

  \caption{Resultados da simulação}
  \label{tab:t2}
\end{table}

O resultado mais importante que foi obtido é que os resultados parecem
ser ligeiramente mais favoráveis do que os do modelo
anterior. Nomeadamente, as médias de pares na fila de espera
diminuíram e o número de transplantes aumentou. Este resultado parece
ser inconsistente com a descrição do modelo, visto que existe uma
maior probabilidade de incompatibilidade. Porém, o que se observa é
que alguns dos pares que eram anteriormente compatíveis entre si
deixaram de o ser, mas vão emparelhar com outros pares da fila que no
modelo anterior ficavam acumulados. Assim, o problema de acumulação de
pares onde era impossível haver emparelhamento é diminuído, causando
as melhorias nos resultados obtidos em relação ao modelo anterior que
compensam a diminuição de compatibilidade.

Novamente os resultados são compatíveis com os parâmetros do enunciado
e são também compatíveis com os do modelo anterior.

%----------------------------------------------------------------------------------------
%	SECTION 3
%----------------------------------------------------------------------------------------

\section{Modelo de Emparelhamento Máximo}
\label{sec:model3}

No modelo de emparelhamento máximo são usados os mesmos métodos base
para introduzir pares na fila e para determinar os tempos em que cada
par chega. A única diferença está no método de emparelhamento que já
não é feito de uma maneira FIFO.

O tempo da simulação é seguido e, quando este passa mais 4 semanas, é
feito um emparelhamento máximo entre os elementos que estão na fila. É
inicialmente construído um grafo onde cada par é representado por um
vértice e há uma aresta entre dois pares caso sejam compatíveis. É
corrido um algoritmo de emparelhamento máximo e os pares que o
algoritmo indica são retirados da fila, sendo feito um transplante
cruzado entre cada dois. É de notar que neste método os elementos de
pares compatíveis entre si continuam a ser imediatamente emparelhados
logo que entrem na fila.

As estatísticas e valores medidos são os mesmos que nas secções
anteriores. Porém, para este modelo são feitas apenas 100 simulações,
visto que o uso do algoritmo de emparelhamento máximo torna cada
simulação menos eficiente e por isso o modelo em geral é mais lento.

Os resultados obtidos por este modelo estão contidos no ficheiro
\texttt{mwm\_simulation.txt}. A Tabela \ref{tab:t3} apresenta um
sumário destes resultados.

\begin{table}[h]
  \centering
  \begin{subtable}[h]{0.3\textwidth}
    \centering
    45
    \caption{Média de pares na fila de espera por semana}
  \end{subtable}
  ~
  \begin{subtable}[h]{0.3\textwidth}
    \centering
    86
    \caption{Média de pares na fila de espera no final da simulação}
  \end{subtable}
  ~
  \begin{subtable}[h]{0.3\textwidth}
    \centering
    $225.84$
    \caption{Número total de transplantes}
  \end{subtable}

  \caption{Resultados da simulação}
  \label{tab:t3}
\end{table}

O resultado mais importante para esta simulação é o facto que os
resultados serem muito próximos dos obtidos no modelo da Secção
\ref{sec:model1}. Isto será devido ao facto que ambos os modelos
emparelham a totalidade dos pares disponíveis, mas que os pares que
são acumulados na fila dominam as estatísticas e são assim comuns aos
dois modelos.

Para verificar a afirmação feita no parágrafo anterior, foi feito um
modelo alternativo onde os pares que são compatíveis com si próprios
não são retirados imediatamente na fila, mas são colocados na fila e
ao fim de 4 semanas se não forem emparelhados com outro par, é feito
um transplante entre membros do mesmo par. Os resultados deste modelo
estão contidos no ficheiro \texttt{mwm\_simulation\_alternative.txt} e
a Tabela \ref{tab:t4} apresenta um sumário dos mesmos.

\begin{table}[h]
  \centering
  \begin{subtable}[h]{0.3\textwidth}
    \centering
    18
    \caption{Média de pares na fila de espera por semana}
  \end{subtable}
  ~
  \begin{subtable}[h]{0.3\textwidth}
    \centering
    27
    \caption{Média de pares na fila de espera no final da simulação}
  \end{subtable}
  ~
  \begin{subtable}[h]{0.3\textwidth}
    \centering
    $283.38$
    \caption{Número total de transplantes}
  \end{subtable}

  \caption{Resultados da simulação}
  \label{tab:t4}
\end{table}

Os resultados foram muito melhores, obtendo menos de metade da média
de pares na fila de espera em ambas as medidas e um aumento no número
de transplantes. Os resultados detalhados ajudam a concluir que de
facto a acumulação de pares na fila diminui, pois, por exemplo, o
número de transplantes de dadores do grupo sanguíneo \texttt{O} para
pacientes do mesmo grupo \texttt{O} aumentou
consideravelmente. Confirma-se assim o \textit{bottleneck} introduzido
pela acumulação descrita.

É de notar que esta simulação alternativa introduz algum erro nos
resultados dos outros modelos em comparação com a situação real, pois
é provável que se um par dador-paciente é compatível entre si, então
quererá efetuar o transplante entre si. O modelo foi construído apenas
para verificar o problema da acumulação de pares.

%----------------------------------------------------------------------------------------
%	SECTION 4
%----------------------------------------------------------------------------------------

\section{Limitações e Hipóteses dos Modelos}
\label{sec:limits}

Terminamos a discussão referindo algumas limitações e hipóteses que
são assumidas pelos modelos. Primeiramente é de referir que a
compatibilidade entre um dador e um paciente é determinada por vários
fatores além dos grupos sanguíneos. Ainda que o modelo da Secção
\ref{sec:model2} aborde esta limitação, continua a ser uma redução da
realidade. Ainda assim, a simulação não deve afastar-se muito dos
valores reais.

Um conjunto de características que foram desprezadas foram questões
ambientais. Por exemplo, a localização geográfica dos elementos de
cada par é um fator importante, pois dever-se-á preferir pares mais
próximos. Mais do que uma preferência, pode haver uma impossibilidade
de deslocação entre distâncias muito grandes, aumentando assim ainda
mais a probabilidade de incompatibilidade.

Uma solução alternativa seria estender os modelos apresentados. Ambos
os modelos fazem emparelhamentos dois a dois, mas poder-se-ia
considerar um modelo que encontra ciclos entre pares de tamanho maior
que dois, otimizando mais os emparelhamentos possíveis (a solução não
iria ser muito diferente da do modelo da Secção \ref{sec:model3},
usando algoritmos de emparelhamento máximo de formas
diferentes). Claro que se introduziriam outros desafios neste modelo,
por exemplo, o tamanho dos ciclos teria de ser limitado (seria pouco
prático considerar um ciclo de transplantes de 100 pessoas, por
exemplo), mas iria melhorar os resultados dos modelos do trabalho.

\section*{Conclusão}

Este trabalho permitiu utilizar métodos de simulação para estudar
características de filas de espera num contexto de transplantes de
rins entre pares de dadores e pacientes. Um problema de grande importância
que pode salvar vidas. Apesar de os resultados obtidos serem coerentes
com os parâmetros indicados, o problema da acumulação de pares deteriorou
os resultados obtidos mas permitiu a sua identificação e tentativa de
melhoramento. Neste tipo de problemas a simulação tem um papel muito
importante pois não é viável fazer as experiências com pacientes reais.

\end{document}
